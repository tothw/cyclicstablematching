\documentclass[letterpaper,12pt,oneside,onecolumn]{article}
\usepackage[margin=1in, bottom=1in, top=1in]{geometry} %1 inch margins
\usepackage{amsmath, amssymb, amstext}
\usepackage{fancyhdr}
\usepackage{algorithm}
\usepackage{algpseudocode}
\usepackage{mathtools}

\DeclarePairedDelimiter{\ceil}{\lceil}{\rceil}
\DeclarePairedDelimiter\floor{\lfloor}{\rfloor}

%Macros
\newcommand{\A}{\mathbb{A}} \newcommand{\C}{\mathbb{C}}
\newcommand{\D}{\mathbb{D}} \newcommand{\F}{\mathbb{F}}
\newcommand{\N}{\mathbb{N}} \newcommand{\R}{\mathbb{R}}
\newcommand{\T}{\mathbb{T}} \newcommand{\Z}{\mathbb{Z}}
\newcommand{\Q}{\mathbb{Q}}
 
 
\newcommand{\cA}{\mathcal{A}} \newcommand{\cB}{\mathcal{B}}
\newcommand{\cC}{\mathcal{C}} \newcommand{\cD}{\mathcal{D}}
\newcommand{\cE}{\mathcal{E}} \newcommand{\cF}{\mathcal{F}}
\newcommand{\cG}{\mathcal{G}} \newcommand{\cH}{\mathcal{H}}
\newcommand{\cI}{\mathcal{I}} \newcommand{\cJ}{\mathcal{J}}
\newcommand{\cK}{\mathcal{K}} \newcommand{\cL}{\mathcal{L}}
\newcommand{\cM}{\mathcal{M}} \newcommand{\cN}{\mathcal{N}}
\newcommand{\cO}{\mathcal{O}} \newcommand{\cP}{\mathcal{P}}
\newcommand{\cQ}{\mathcal{Q}} \newcommand{\cR}{\mathcal{R}}
\newcommand{\cS}{\mathcal{S}} \newcommand{\cT}{\mathcal{T}}
\newcommand{\cU}{\mathcal{U}} \newcommand{\cV}{\mathcal{V}}
\newcommand{\cW}{\mathcal{W}} \newcommand{\cX}{\mathcal{X}}
\newcommand{\cY}{\mathcal{Y}} \newcommand{\cZ}{\mathcal{Z}}

\newcommand\numberthis{\addtocounter{equation}{1}\tag{\theequation}}
%Page style
\pagestyle{fancy}

\listfiles

\raggedbottom

\rhead{William Justin Toth Cyclic Stable Matching Enumeration} %CHANGE n to ASSIGNMENT NUMBER ijk TO COURSE CODE
\renewcommand{\headrulewidth}{1pt} %heading underlined
%\renewcommand{\baselinestretch}{1.2} % 1.2 line spacing for legibility (optional)

\begin{document}
\section{Introduction}
\paragraph{}
In the cyclic stable matching problem we are given three sets $A$, $B$, and $C$ of equal size $n$. The elements of $A$ each maintain a total preference order over the elements of $B$ which we will denote by $<_a$ for each $a \in A$. Similarly elements of $B$ maintain a total preference order over elements of $C$ and elements of $C$ maintain a total preference order over elements of $A$. We define a matching $M$ to be a set of the form
$$\{ \{a,b,c\}: a \in A, b \in B, c\in C\}$$
satisfying 
$$|\{ \{a,b,c\} \in M : v \in \{a,b,c\}\}| \leq 1$$
for all $v \in A \cup B \cup C$. For each $a \in A$ we define $M(a) = b$ such that $\{a,b,c\} \in M$ for some $c$, and $M(a) = 0$ ($0$ is simply some empty element) if no such $b$ exists. We define $M(b)$ for $b \in B$ and $M(c)$ for $c \in C$ analogously. By the definition of $M$ it is clear that these are well defined.
\paragraph{}
We call a matching stable if there does not exist a triple $(a,b,c) \in A \times B \times C$ such that 
$$ b >_a M(a) \quad\text{and}\quad c >_b M(b) \quad\text{and}\quad a>_c M(c).$$
Herein we adopt the convention that any element is preferable to $0$ in any ordering. It is conjectured  by Knuth that a stable matching exists for any instance of the cyclic stable matching problem. It is known that this result holds for $n \leq 4$. We propose a computational search strategy to decide if the conjecture holds for higher $n$, say $n=5$.
\section{}
\end{document}
